\documentclass{article}
\usepackage[utf8]{inputenc}
\usepackage[spanish]{babel}
\usepackage{algorithm}
\usepackage{algorithmic}
\usepackage{amsmath}
\usepackage{geometry}
\geometry{margin=1in}

\title{Pseudocódigo: Problema de Selección de Actividades con Fuerza Bruta}
\author{Algoritmo de Fuerza Bruta}
\date{\today}

\begin{document}

\maketitle

\section{Descripción del Problema}
%<*enunciado>
El problema de selección de actividades consiste en seleccionar el máximo número de actividades mutuamente compatibles de un conjunto dado. Cada actividad tiene un tiempo de inicio y un tiempo de finalización. Dos actividades son compatibles si no se solapan en el tiempo (la finalización de una es menor o igual al inicio de la otra). El objetivo es encontrar el subconjunto de máximo tamaño que no contenga actividades que se solapen.
%</enunciado>

\section{Pseudocódigo}

\begin{algorithm}
\caption{Selección de Actividades con Fuerza Bruta}
\begin{algorithmic}[1]
%<*codigo1a>
\REQUIRE Un arreglo $activities$ de $n$ elementos donde cada elemento es una tupla $(start, end)$
\ENSURE El subconjunto de máximo tamaño de actividades mutuamente compatibles

\STATE $n \leftarrow$ longitud de $activities$
\STATE $best \leftarrow$ lista vacía

\COMMENT{Probar todos los subconjuntos posibles}
\FOR{$r = 1$ \textbf{to} $n$}
    \FOR{cada $subset$ en $combinations(activities, r)$}
        \IF{$is\_valid(subset)$ $\land$ $|subset| > |best|$}
            \STATE $best \leftarrow subset$
        \ENDIF
    \ENDFOR
\ENDFOR
\RETURN $best$
%</codigo1a>
\end{algorithmic}
\end{algorithm}

\begin{algorithm}
\caption{Función de Validación}
\begin{algorithmic}[1]
%<*codigo1b>
\REQUIRE Un subconjunto $schedule$ de actividades
\ENSURE Verdadero si las actividades no se solapan, Falso en caso contrario

\FOR{$i = 0$ \textbf{to} $|schedule|-1$}
    \FOR{$j = i + 1$ \textbf{to} $|schedule|-1$}
        \IF{$\neg\big( (schedule[i].end \\leq schedule[j].start) \\\lor (schedule[j].end \\leq schedule[i].start) \big)$}
            \RETURN false
        \ENDIF
    \ENDFOR
\ENDFOR
\RETURN true
%</codigo1b>
\end{algorithmic}
\end{algorithm}

\section{Explicación del Algoritmo}

\subsection{Enfoque de Fuerza Bruta}
%<*exp1>
El algoritmo de fuerza bruta para el problema de selección de actividades funciona de la siguiente manera:

\begin{itemize}
    \item \textbf{Generación de subconjuntos:} Se generan todos los posibles subconjuntos de actividades usando la función $combinations$
    \item \textbf{Validación:} Para cada subconjunto generado, se verifica si las actividades son mutuamente compatibles
    \item \textbf{Optimización:} Se mantiene el mejor subconjunto encontrado hasta el momento
\end{itemize}

La validación de compatibilidad se basa en que dos actividades $(start_i, end_i)$ y $(start_j, end_j)$ son compatibles si:
$$end_i \leq start_j \text{ o } end_j \leq start_i$$
%</exp1>

\subsection{Complejidad}
%<*exp2>
\begin{itemize}
    \item \textbf{Tiempo:} $O(2^n \cdot n^2)$
    \begin{itemize}
        \item $O(2^n)$ para generar todos los subconjuntos posibles
        \item $O(n^2)$ para validar cada subconjunto (verificar solapamientos)
    \end{itemize}
    \item \textbf{Espacio:} $O(n)$ para almacenar el mejor subconjunto encontrado
\end{itemize}
%</exp2>

\section{Resolución Paso a Paso}

\subsection{Ejemplo: Actividades = [(1, 3), (2, 5), (4, 6)]}

\textbf{Paso 1:} Identificar todas las actividades
\begin{itemize}
    \item Actividad A: (1, 3) - inicia en 1, termina en 3
    \item Actividad B: (2, 5) - inicia en 2, termina en 5  
    \item Actividad C: (4, 6) - inicia en 4, termina en 6
\end{itemize}

\textbf{Paso 2:} Generar todos los subconjuntos de tamaño 1
\begin{itemize}
    \item \{A\} = \{(1, 3)\} - Válido (1 actividad)
    \item \{B\} = \{(2, 5)\} - Válido (1 actividad)
    \item \{C\} = \{(4, 6)\} - Válido (1 actividad)
\end{itemize}
Mejor hasta ahora: cualquier subconjunto de tamaño 1

\textbf{Paso 3:} Generar todos los subconjuntos de tamaño 2
\begin{itemize}
    \item \{A, B\} = \{(1, 3), (2, 5)\}
    \begin{itemize}
        \item ¿A y B se solapan? A termina en 3, B inicia en 2
        \item Como 3 > 2, se solapan. NO es válido.
    \end{itemize}
    \item \{A, C\} = \{(1, 3), (4, 6)\}
    \begin{itemize}
        \item ¿A y C se solapan? A termina en 3, C inicia en 4
        \item Como $3 \le 4$, NO se solapan. ES válido.
    \end{itemize}
    \item \{B, C\} = \{(2, 5), (4, 6)\}
    \begin{itemize}
        \item ¿B y C se solapan? B termina en 5, C inicia en 4
        \item Como 5 > 4, se solapan. NO es válido.
    \end{itemize}
\end{itemize}
Mejor hasta ahora: \{A, C\} = \{(1, 3), (4, 6)\} (tamaño 2)

\textbf{Paso 4:} Generar subconjunto de tamaño 3
\begin{itemize}
    \item \{A, B, C\} = \{(1, 3), (2, 5), (4, 6)\}
    \begin{itemize}
        \item Verificar solapamientos:
        \begin{itemize}
            \item A y B: $3 > 2$ $\Rightarrow$ Se solapan
            \item A y C: $3 \le 4$ $\Rightarrow$ NO se solapan
            \item B y C: $5 > 4$ $\Rightarrow$ Se solapan
        \end{itemize}
        \item Como hay al menos un solapamiento, NO es válido.
    \end{itemize}
\end{itemize}

\textbf{Resultado Final:} El subconjunto óptimo es \{A, C\} = \{(1, 3), (4, 6)\} con 2 actividades.

\subsection{Visualización Temporal}
\begin{verbatim}
Tiempo: 0  1  2  3  4  5  6  7
A:      |-----|
B:         |--------|
C:            |-----|

Solución óptima: A y C (no se solapan)
\end{verbatim}

\section{Análisis de Complejidad}

\subsection{Comparación con Algoritmo Greedy}
%<*exp3>
\begin{itemize}
    \item \textbf{Fuerza Bruta:} $O(2^n \cdot n^2)$ - Encuentra la solución óptima pero es exponencial
    \item \textbf{Algoritmo Greedy:} $O(n \log n)$ - Solución óptima cuando se ordena por tiempo de finalización
\end{itemize}

El algoritmo de fuerza bruta es útil para:
\begin{itemize}
    \item Verificar la corrección de algoritmos más eficientes
    \item Problemas pequeños donde la complejidad exponencial es manejable
    \item Casos donde se necesitan múltiples criterios de optimización
\end{itemize}
%</exp3>

\section{Conclusiones}

El algoritmo de fuerza bruta para el problema de selección de actividades garantiza encontrar la solución óptima pero tiene una complejidad exponencial. Es apropiado para casos de estudio y verificación, pero para problemas reales se recomienda usar el algoritmo greedy que tiene complejidad $O(n \log n)$ y produce resultados óptimos cuando se ordena por tiempo de finalización.

\end{document}
