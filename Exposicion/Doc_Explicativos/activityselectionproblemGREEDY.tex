\documentclass{article}
\usepackage[utf8]{inputenc}
\usepackage[spanish]{babel}
\usepackage{algorithm}
\usepackage{algorithmic}
\usepackage{amsmath}
\usepackage{geometry}
\geometry{margin=1in}

\title{Pseudocódigo: Problema de Selección de Actividades con Algoritmo Greedy}
\author{Algoritmo Greedy}
\date{\today}

\begin{document}

\maketitle

\section{Descripción del Problema}
%<*enunciado>
El problema de selección de actividades consiste en seleccionar el máximo número de actividades mutuamente compatibles de un conjunto dado. Cada actividad tiene un tiempo de inicio y un tiempo de finalización. Dos actividades son compatibles si no se solapan en el tiempo (la finalización de una es menor o igual al inicio de la otra). El objetivo es encontrar el subconjunto de máximo tamaño que no contenga actividades que se solapen.

El algoritmo greedy resuelve este problema de manera óptima ordenando las actividades por su tiempo de finalización y seleccionando de manera greedy.
%</enunciado>

\section{Pseudocódigo}

\begin{algorithm}
\caption{Selección de Actividades con Algoritmo Greedy}
\begin{algorithmic}[1]
%<*codigo>
\REQUIRE Arreglos $start$ y $finish$ de $n$ elementos con tiempos de inicio y finalización
\ENSURE El número máximo de actividades mutuamente compatibles

\STATE $activities \leftarrow$ crear lista de tuplas $(start[i], finish[i])$ para $i = 0$ a $n-1$
\STATE Ordenar $activities$ por tiempo de finalización (segundo elemento de cada tupla)
\STATE $count \leftarrow 1$ \COMMENT{Al menos una actividad puede realizarse}
\STATE $j \leftarrow 0$ \COMMENT{Índice de la última actividad seleccionada}

\FOR{$i = 1$ \TO $n-1$}
    \IF{$activities[i].start > activities[j].finish$}
        \STATE $count \leftarrow count + 1$
        \STATE $j \leftarrow i$ \COMMENT{Actualizar última actividad seleccionada}
    \ENDIF
\ENDFOR
\RETURN $count$
%</codigo>
\end{algorithmic}
\end{algorithm}

\section{Explicación del Algoritmo}

\subsection{Estrategia Greedy}
%<*exp1>
El algoritmo greedy para el problema de selección de actividades utiliza la siguiente estrategia:

\begin{itemize}
    \item \textbf{Ordenamiento:} Ordena todas las actividades por su tiempo de finalización en orden creciente
    \item \textbf{Selección Greedy:} Siempre selecciona la primera actividad que no se solape con la última actividad seleccionada
    \item \textbf{Justificación:} Al seleccionar la actividad que termina primero, se maximiza el tiempo disponible para actividades futuras
\end{itemize}

La elección greedy es óptima porque:
\begin{itemize}
    \item Si existe una solución óptima que no incluye la actividad que termina primero, se puede reemplazar la primera actividad de esa solución por la que termina primero
    \item Esto no empeora la solución y mantiene la optimalidad
\end{itemize}
%</exp1>

\subsection{Subestructura Óptima}
%<*exp2>
Para el problema de selección de actividades, la subestructura óptima se cumple:

Si $S$ es una solución óptima para el conjunto de actividades $A$, y $a_1$ es la actividad que termina primero en $A$, entonces:
\begin{itemize}
    \item $a_1$ debe estar en alguna solución óptima
    \item El resto de la solución óptima debe ser óptima para el conjunto $A'$ de actividades que no se solapan con $a_1$
\end{itemize}

Esto se debe a que seleccionar $a_1$ no interfiere con las opciones futuras, sino que las maximiza.
%</exp2>

\subsection{Complejidad}
%<*exp3>
\begin{itemize}
    \item \textbf{Tiempo:} $O(n \log n)$
    \begin{itemize}
        \item $O(n \log n)$ para ordenar las actividades por tiempo de finalización
        \item $O(n)$ para recorrer las actividades y hacer las selecciones
    \end{itemize}
    \item \textbf{Espacio:} $O(n)$ para almacenar las actividades ordenadas
\end{itemize}
%</exp3>

\section{Resolución Paso a Paso}

\subsection{Ejemplo}
%<*ejemplo>
\textbf{Actividades = [(1, 2), (3, 4), (0, 6), (5, 7), (8, 9), (5, 9)]}
\textbf{Paso 1:} Crear tuplas de actividades
\begin{itemize}
    \item Actividad 0: (1, 2)
    \item Actividad 1: (3, 4) 
    \item Actividad 2: (0, 6)
    \item Actividad 3: (5, 7)
    \item Actividad 4: (8, 9)
    \item Actividad 5: (5, 9)
\end{itemize}

\textbf{Paso 2:} Ordenar por tiempo de finalización
\begin{itemize}
    \item Actividad 0: (1, 2) - termina en 2
    \item Actividad 1: (3, 4) - termina en 4
    \item Actividad 2: (0, 6) - termina en 6
    \item Actividad 3: (5, 7) - termina en 7
    \item Actividad 4: (8, 9) - termina en 9
    \item Actividad 5: (5, 9) - termina en 9
\end{itemize}

\textbf{Paso 3:} Aplicar algoritmo greedy

Inicialización:
\begin{itemize}
    \item $count = 1$ (siempre se puede hacer al menos una actividad)
    \item $j = 0$ (primera actividad seleccionada: Actividad 0)
\end{itemize}

Iteraciones:
\begin{itemize}
    \item \textbf{i = 1:} Actividad 1 (3, 4)
    \begin{itemize}
        \item ¿$activities[1].start > activities[0].finish$?
        \item ¿$3 > 2$? SÍ
        \item Seleccionar Actividad 1
        \item $count = 2$, $j = 1$
    \end{itemize}
    
    \item \textbf{i = 2:} Actividad 2 (0, 6)
    \begin{itemize}
        \item ¿$activities[2].start > activities[1].finish$?
        \item ¿$0 > 4$? NO
        \item NO seleccionar Actividad 2 (se solapa)
    \end{itemize}
    
    \item \textbf{i = 3:} Actividad 3 (5, 7)
    \begin{itemize}
        \item ¿$activities[3].start > activities[1].finish$?
        \item ¿$5 > 4$? SÍ
        \item Seleccionar Actividad 3
        \item $count = 3$, $j = 3$
    \end{itemize}
    
    \item \textbf{i = 4:} Actividad 4 (8, 9)
    \begin{itemize}
        \item ¿$activities[4].start > activities[3].finish$?
        \item ¿$8 > 7$? SÍ
        \item Seleccionar Actividad 4
        \item $count = 4$, $j = 4$
    \end{itemize}
    
    \item \textbf{i = 5:} Actividad 5 (5, 9)
    \begin{itemize}
        \item ¿$activities[5].start > activities[4].finish$?
        \item ¿$5 > 9$? NO
        \item NO seleccionar Actividad 5 (se solapa)
    \end{itemize}
\end{itemize}

\textbf{Resultado Final:} Se pueden realizar 4 actividades: Actividad 0, Actividad 1, Actividad 3, y Actividad 4.
%</ejemplo>
\subsection{Visualización Temporal}
%<*visualizacion>
\begin{verbatim}
Tiempo: 0  1  2  3  4  5  6  7  8  9  10
Act 0:     |--|                           [OK] Seleccionada
Act 1:        |--|                        [OK] Seleccionada  
Act 2:  |----------|                      [X] No seleccionada (se solapa)
Act 3:           |--|                     [OK] Seleccionada
Act 4:                |--|                [OK] Seleccionada
Act 5:           |-------|                [X] No seleccionada (se solapa)

Solución óptima: 4 actividades
\end{verbatim}
%</visualizacion>
\section{Demostración de Optimalidad}

\subsection{Teorema}
%<*exp4>
El algoritmo greedy para selección de actividades produce una solución óptima.

\textbf{Demostración por contradicción:}

Supongamos que el algoritmo greedy no produce una solución óptima. Sea $G$ la solución del algoritmo greedy y $O$ una solución óptima con $|O| > |G|$.

Sea $a_i$ la primera actividad en $O$ que no está en $G$. Como el algoritmo greedy siempre selecciona la actividad que termina primero que no se solapa, y $a_i$ no está en $G$, debe ser que $a_i$ se solapa con alguna actividad en $G$ que termina antes que $a_i$.

Pero esto es imposible porque $a_i$ es parte de $O$, que es una solución válida. Por lo tanto, $a_i$ no puede solaparse con actividades en $G$ que terminen antes.

Esta contradicción demuestra que $|O| \leq |G|$, por lo que $G$ es óptima.
%</exp4>

\section{Comparación con Fuerza Bruta}

\begin{itemize}
    \item \textbf{Complejidad:}
    \begin{itemize}
        \item Greedy: $O(n \log n)$
        \item Fuerza Bruta: $O(2^n \cdot n^2)$
    \end{itemize}
    \item \textbf{Ventajas del Greedy:}
    \begin{itemize}
        \item Mucho más eficiente para problemas grandes
        \item Fácil de implementar
        \item Produce solución óptima
    \end{itemize}
    \item \textbf{Cuándo usar cada uno:}
    \begin{itemize}
        \item Greedy: Para problemas reales y grandes
        \item Fuerza Bruta: Para verificación y problemas pequeños
    \end{itemize}
\end{itemize}

\section{Conclusiones}

El algoritmo greedy para selección de actividades es una solución elegante y eficiente que produce resultados óptimos en tiempo $O(n \log n)$. Su fortaleza radica en la elección inteligente del criterio de ordenamiento (por tiempo de finalización) y la estrategia de selección greedy, que garantiza la optimalidad de la solución.

Este algoritmo demuestra el poder de las estrategias greedy cuando se aplican correctamente a problemas que exhiben subestructura óptima y la propiedad de elección greedy.

\end{document}
