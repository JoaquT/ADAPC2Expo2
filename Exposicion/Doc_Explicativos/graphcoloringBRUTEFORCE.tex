\documentclass{article}
\usepackage[utf8]{inputenc}
\usepackage[spanish]{babel}
\usepackage{algorithm}
\usepackage{algorithmic}
\usepackage{amsmath}
\usepackage{geometry}
\usepackage{tikz}
\usepackage{xcolor}
\geometry{margin=1in}

\title{Pseudocódigo: Coloreo de Grafos con Fuerza Bruta}
\author{Algoritmo de Fuerza Bruta}
\date{\today}

\begin{document}

\maketitle

\section{Descripción del Problema}
%<*enunciado>
El problema de coloreo de grafos consiste en asignar colores a los vértices de un grafo de manera que ningún par de vértices adyacentes compartan el mismo color. El objetivo es determinar si es posible colorear un grafo dado usando exactamente $k$ colores, donde $k$ es un número entero positivo.

El algoritmo de fuerza bruta prueba todas las posibles asignaciones de colores hasta encontrar una válida o determinar que no existe solución con $k$ colores.
%</enunciado>

\section{Pseudocódigo}

\begin{algorithm}
\caption{Coloreo de Grafos con Fuerza Bruta}
\begin{algorithmic}[1]
%<*codigo1a>
\REQUIRE Un grafo $G$ representado como lista de adyacencia y un entero $k$
\ENSURE Una asignación válida de colores o \textbf{null} si no existe solución

\STATE $n \leftarrow$ número de vértices en $G$
\FOR{cada $assignment$ en $product(range(1, k+1), repeat=n)$}
    \IF{$is\_valid(G, assignment)$}
        \RETURN $assignment$
    \ENDIF
\ENDFOR
\RETURN \textbf{null} \COMMENT{No existe coloreo válido con $k$ colores}
%</codigo1a>
\end{algorithmic}
\end{algorithm}

\begin{algorithm}
\caption{Función de Validación de Coloreo}
\begin{algorithmic}[1]
%<*codigo1b>
\REQUIRE Un grafo $G$ y una asignación de colores $colors$
\ENSURE Verdadero si la asignación es válida, Falso en caso contrario

\FOR{$u = 0$ \textbf{to} $|G|-1$}
    \FOR{cada $v$ en $G[u]$}
        \IF{$colors[u] = colors[v]$}
            \RETURN false \COMMENT{Vértices adyacentes tienen el mismo color}
        \ENDIF
    \ENDFOR
\ENDFOR
\RETURN true
%</codigo1b>
\end{algorithmic}
\end{algorithm}

\section{Explicación del Algoritmo}

\subsection{Enfoque de Fuerza Bruta}
%<*exp1>
El algoritmo de fuerza bruta para coloreo de grafos funciona de la siguiente manera:

\begin{itemize}
    \item \textbf{Generación de asignaciones:} Utiliza el producto cartesiano para generar todas las posibles asignaciones de $k$ colores a $n$ vértices
    \item \textbf{Validación:} Para cada asignación generada, verifica si cumple la restricción de coloreo (vértices adyacentes no pueden tener el mismo color)
    \item \textbf{Búsqueda exhaustiva:} Prueba sistemáticamente todas las combinaciones hasta encontrar una válida
\end{itemize}

El número total de asignaciones posibles es $k^n$, donde cada vértice puede recibir cualquiera de los $k$ colores disponibles.
%</exp1>

\subsection{Condición de Validación}
%<*exp2>
Una asignación de colores es válida si y solo si:
$$\forall (u,v) \in E: color[u] \neq color[v]$$

Donde:
\begin{itemize}
    \item $E$ es el conjunto de aristas del grafo
    \item $color[u]$ es el color asignado al vértice $u$
    \item La condición garantiza que vértices adyacentes tengan colores diferentes
\end{itemize}
%</exp2>

\subsection{Complejidad}
%<*exp3>
\begin{itemize}
    \item \textbf{Tiempo:} $O(k^n \cdot n \cdot m)$
    \begin{itemize}
        \item $O(k^n)$ para generar todas las asignaciones posibles
        \item $O(n \cdot m)$ para validar cada asignación, donde $m$ es el número de aristas
    \end{itemize}
    \item \textbf{Espacio:} $O(n)$ para almacenar la asignación actual
\end{itemize}
%</exp3>

\section{Resolución Paso a Paso}

\subsection{Ejemplo: Grafo con 6 vértices y k = 3 colores}

\textbf{Grafo de entrada:}
\begin{itemize}
    \item Vértice 0: conectado a [1, 5, 2]
    \item Vértice 1: conectado a [0, 2]
    \item Vértice 2: conectado a [1, 3, 0, 5]
    \item Vértice 3: conectado a [2, 4]
    \item Vértice 4: conectado a [3, 5]
    \item Vértice 5: conectado a [0, 4, 2]
\end{itemize}

\textbf{Representación visual del grafo:}
\begin{center}
\begin{tikzpicture}[node distance=2cm]
    \node[circle, draw, fill=blue!20] (0) at (0,0) {0};
    \node[circle, draw, fill=red!20] (1) at (2,0) {1};
    \node[circle, draw, fill=green!20] (2) at (1,1.5) {2};
    \node[circle, draw, fill=blue!20] (3) at (3,1.5) {3};
    \node[circle, draw, fill=red!20] (4) at (4,0) {4};
    \node[circle, draw, fill=green!20] (5) at (2,-1.5) {5};
    
    \draw (0) -- (1);
    \draw (0) -- (2);
    \draw (0) -- (5);
    \draw (1) -- (2);
    \draw (2) -- (3);
    \draw (2) -- (5);
    \draw (3) -- (4);
    \draw (4) -- (5);
\end{tikzpicture}
\end{center}

\textbf{Paso 1:} Identificar parámetros del problema
\begin{itemize}
    \item Número de vértices: $n = 6$
    \item Número de colores: $k = 3$ (colores 1, 2, 3)
    \item Total de asignaciones posibles: $3^6 = 729$
\end{itemize}

\textbf{Paso 2:} Ejemplos de asignaciones inválidas (primeras iteraciones)

\begin{itemize}
    \item \textbf{Asignación 1:} [1, 1, 1, 1, 1, 1]
    \begin{itemize}
        \item Todos los vértices tienen color 1
        \item Vértices adyacentes (0,1), (0,2), (0,5) tienen el mismo color
        \item \textbf{Resultado:} INVÁLIDA
    \end{itemize}
    
    \item \textbf{Asignación 2:} [1, 1, 1, 1, 1, 2]
    \begin{itemize}
        \item Vértices 0-4 tienen color 1, vértice 5 tiene color 2
        \item Vértices adyacentes (0,1), (0,2) tienen el mismo color
        \item \textbf{Resultado:} INVÁLIDA
    \end{itemize}
    
    \item \textbf{Asignación 3:} [1, 1, 1, 1, 1, 3]
    \begin{itemize}
        \item Vértices 0-4 tienen color 1, vértice 5 tiene color 3
        \item Vértices adyacentes (0,1), (0,2) tienen el mismo color
        \item \textbf{Resultado:} INVÁLIDA
    \end{itemize}
\end{itemize}

\textbf{Paso 3:} Proceso de búsqueda sistemática

El algoritmo continúa probando asignaciones en orden lexicográfico:
\begin{itemize}
    \item [1,1,1,1,2,1] - Inválida
    \item [1,1,1,1,2,2] - Inválida
    \item [1,1,1,1,2,3] - Inválida
    \item [1,1,1,1,3,1] - Inválida
    \item ... (continúa hasta encontrar una válida)
\end{itemize}

\textbf{Paso 4:} Encontrar la primera asignación válida

Después de probar múltiples asignaciones, el algoritmo encuentra:
\begin{itemize}
    \item \textbf{Asignación válida:} [1, 2, 3, 1, 2, 3]
    \begin{itemize}
        \item Vértice 0: color 1
        \item Vértice 1: color 2
        \item Vértice 2: color 3
        \item Vértice 3: color 1
        \item Vértice 4: color 2
        \item Vértice 5: color 3
    \end{itemize}
\end{itemize}

\textbf{Verificación de validez:}
\begin{itemize}
    \item (0,1): color 1 $\neq$ color 2 [OK]
    \item (0,2): color 1 $\neq$ color 3 [OK]
    \item (0,5): color 1 $\neq$ color 3 [OK]
    \item (1,2): color 2 $\neq$ color 3 [OK]
    \item (2,3): color 3 $\neq$ color 1 [OK]
    \item (2,5): color 3 $\neq$ color 3 [X]
\end{itemize}

\textbf{Corrección:} La asignación [1,2,3,1,2,3] no es válida porque los vértices 2 y 5 (que son adyacentes) tienen el mismo color.

\textbf{Paso 5:} Continuar búsqueda hasta encontrar solución válida

El algoritmo encuentra finalmente:
\begin{itemize}
    \item \textbf{Solución válida:} [1, 2, 1, 2, 3, 2]
    \begin{itemize}
        \item Vértice 0: color 1
        \item Vértice 1: color 2
        \item Vértice 2: color 1
        \item Vértice 3: color 2
        \item Vértice 4: color 3
        \item Vértice 5: color 2
    \end{itemize}
\end{itemize}

\textbf{Verificación completa:}
\begin{itemize}
    \item (0,1): color 1 $\neq$ color 2 [OK]
    \item (0,2): color 1 $\neq$ color 1 [X]
\end{itemize}

\textbf{Corrección final:} La solución válida es [1, 2, 3, 1, 2, 1]:
\begin{itemize}
    \item Vértice 0: color 1
    \item Vértice 1: color 2
    \item Vértice 2: color 3
    \item Vértice 3: color 1
    \item Vértice 4: color 2
    \item Vértice 5: color 1
\end{itemize}

\subsection{Visualización del Coloreo Final}
\begin{center}
\begin{tikzpicture}[node distance=2cm]
    \node[circle, draw, fill=red!50] (0) at (0,0) {0 (1)};
    \node[circle, draw, fill=blue!50] (1) at (2,0) {1 (2)};
    \node[circle, draw, fill=green!50] (2) at (1,1.5) {2 (3)};
    \node[circle, draw, fill=red!50] (3) at (3,1.5) {3 (1)};
    \node[circle, draw, fill=blue!50] (4) at (4,0) {4 (2)};
    \node[circle, draw, fill=red!50] (5) at (2,-1.5) {5 (1)};
    
    \draw (0) -- (1);
    \draw (0) -- (2);
    \draw (0) -- (5);
    \draw (1) -- (2);
    \draw (2) -- (3);
    \draw (2) -- (5);
    \draw (3) -- (4);
    \draw (4) -- (5);
\end{tikzpicture}
\end{center}

\textbf{Resultado Final:} Coloreo válido encontrado: [1, 2, 3, 1, 2, 1] usando 3 colores.

\section{Análisis de Complejidad}

\subsection{Comparación con Otros Enfoques}
%<*exp4>
\begin{itemize}
    \item \textbf{Fuerza Bruta:} $O(k^n \cdot n \cdot m)$ - Garantiza encontrar solución si existe
    \item \textbf{Algoritmo Greedy:} $O(n^2)$ - Rápido pero puede no encontrar solución óptima
    \item \textbf{Backtracking:} $O(k^n)$ en el peor caso, pero más eficiente en promedio
\end{itemize}

\textbf{Ventajas del enfoque de fuerza bruta:}
\begin{itemize}
    \item Simplicidad de implementación
    \item Garantía de encontrar solución si existe
    \item Útil para grafos pequeños y verificación
\end{itemize}

\textbf{Desventajas:}
\begin{itemize}
    \item Complejidad exponencial
    \item Ineficiente para grafos grandes
    \item No aprovecha heurísticas o podas
\end{itemize}
%</exp4>

\section{Conclusiones}

El algoritmo de fuerza bruta para coloreo de grafos es un enfoque directo que garantiza encontrar una solución válida si existe, pero su complejidad exponencial lo hace impráctico para grafos grandes. Es útil para:

\begin{itemize}
    \item Verificar la corrección de algoritmos más sofisticados
    \item Resolver instancias pequeñas del problema
    \item Estudios teóricos sobre la complejidad del problema
\end{itemize}

Para aplicaciones prácticas, se recomienda usar algoritmos más eficientes como backtracking con podas o algoritmos heurísticos específicos para el problema de coloreo de grafos.

\end{document}
