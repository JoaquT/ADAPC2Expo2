\documentclass{article}
\usepackage[utf8]{inputenc}
\usepackage[spanish]{babel}
\usepackage{algorithm}
\usepackage{algorithmic}
\usepackage{amsmath}
\usepackage{geometry}
\usepackage{tikz}
\usepackage{xcolor}
\providecommand{\abs}[1]{\lvert#1\rvert}
\providecommand{\norm}[1]{\lVert#1\rVert}
\geometry{margin=1in}

\title{Pseudocódigo: Coloreo de Grafos con Algoritmo Greedy}
\author{Algoritmo Greedy}
\date{\today}

\begin{document}

\maketitle

\section{Descripción del Problema}
%<*enunciado>
El problema de coloreo de grafos consiste en asignar colores a los vértices de un grafo de manera que ningún par de vértices adyacentes compartan el mismo color. El objetivo es encontrar una coloración válida usando el menor número posible de colores.

El algoritmo greedy resuelve este problema asignando a cada vértice el menor color disponible que no esté siendo usado por sus vértices adyacentes. Aunque no garantiza usar el número mínimo de colores (número cromático), proporciona una coloración válida de manera eficiente.
%</enunciado>

\section{Pseudocódigo}

\begin{algorithm}
\caption{Coloreo de Grafos con Algoritmo Greedy}
\begin{algorithmic}[1]
%<*codigo1a>
\REQUIRE Un grafo $G$ representado como lista de adyacencia con $V$ vértices
\ENSURE Una asignación válida de colores para todos los vértices

\STATE $result \leftarrow$ arreglo de tamaño $V$ inicializado en -1
\STATE $result[0] \leftarrow 0$ \COMMENT{Asignar primer color al primer vértice}

\STATE $available \leftarrow$ arreglo booleano de tamaño $V$ inicializado en false

\COMMENT{Asignar colores a los vértices restantes}
\FOR{$u = 1$ \TO $V-1$}
    \COMMENT{Marcar colores de vértices adyacentes como no disponibles}

    \FOR{$i=0$ \TO $|adj[u]|-1$}
        \IF{$result[adj[u][i]] \neq -1$}
            \STATE $available[result[adj[u][i]]] \leftarrow$ true
        \ENDIF
    \ENDFOR
    
    \COMMENT{Encontrar el primer color disponible}
    \STATE $cr \leftarrow 0$
    \WHILE{$cr < V$}
        \IF{$available[cr] = false$}
            \STATE \textbf{break}
        \ENDIF
        \STATE $cr \leftarrow cr + 1$
    \ENDWHILE
    
    \COMMENT{Asignar el color encontrado}
    \STATE $result[u] \leftarrow cr$
    
    \COMMENT{Resetear valores para la siguiente iteración}
    \FOR{$i = 0$ \TO $|adj[u]|-1$}
        \IF{$result[adj[u][i]] \neq -1$}
            \STATE $available[result[adj[u][i]]] \leftarrow$ false
        \ENDIF
    \ENDFOR
\ENDFOR
\RETURN $result$
%</codigo1a>
\end{algorithmic}
\end{algorithm}

\section{Explicación del Algoritmo}

\subsection{Estrategia Greedy}
%<*exp1>
El algoritmo greedy para coloreo de grafos utiliza la siguiente estrategia:

\begin{itemize}
    \item \textbf{Orden fijo:} Procesa los vértices en un orden predeterminado (0, 1, 2, ..., V-1)
    \item \textbf{Selección greedy:} Para cada vértice, asigna el menor color disponible que no esté siendo usado por sus vecinos
    \item \textbf{Disponibilidad de colores:} Mantiene un arreglo que indica qué colores están siendo usados por vértices adyacentes
\end{itemize}

La elección greedy de usar el menor color disponible es una heurística que, aunque no garantiza optimalidad, tiende a producir buenas coloraciones en la práctica.
%</exp1>

\subsection{Proceso de Asignación}
%<*exp2>
Para cada vértice $u$ (excepto el primero), el algoritmo:

\begin{enumerate}
    \item \textbf{Marca colores no disponibles:} Recorre todos los vértices adyacentes a $u$ y marca sus colores como no disponibles
    \item \textbf{Busca primer color disponible:} Encuentra el menor color (índice) que no esté marcado como no disponible
    \item \textbf{Asigna el color:} Asigna el color encontrado al vértice $u$
    \item \textbf{Resetea disponibilidad:} Limpia las marcas de disponibilidad para el siguiente vértice
\end{enumerate}

Este proceso garantiza que cada vértice reciba un color válido (diferente al de sus vecinos).
%</exp2>

\subsection{Complejidad}
%<*exp3>
\begin{itemize}
    \item \textbf{Tiempo:} $O(V + E)$ donde $V$ es el número de vértices y $E$ el número de aristas
    \begin{itemize}
        \item $O(V)$ para procesar cada vértice
        \item $O(E)$ para procesar todas las aristas al verificar adyacencias
        \item El bucle interno para encontrar el primer color disponible es $O(V)$ en el peor caso
    \end{itemize}
    \item \textbf{Espacio:} $O(V)$ para almacenar el resultado y el arreglo de disponibilidad
\end{itemize}

La complejidad total es $O(V^2 + E)$ en el peor caso, pero típicamente es mucho mejor en grafos dispersos.
%</exp3>

\section{Resolución Paso a Paso}

\subsection{Ejemplo 1: Grafo 1 con 5 vértices}

\textbf{Grafo de entrada:}
\begin{itemize}
    \item Vértice 0: conectado a [1, 2]
    \item Vértice 1: conectado a [0, 2, 3]
    \item Vértice 2: conectado a [0, 1, 3]
    \item Vértice 3: conectado a [1, 2, 4]
    \item Vértice 4: conectado a [3]
\end{itemize}

\textbf{Representación visual del grafo:}
\begin{center}
\begin{tikzpicture}[node distance=2cm]
    \node[circle, draw, fill=red!50] (0) at (0,0) {0};
    \node[circle, draw, fill=blue!50] (1) at (2,0) {1};
    \node[circle, draw, fill=green!50] (2) at (1,1.5) {2};
    \node[circle, draw, fill=red!50] (3) at (3,1.5) {3};
    \node[circle, draw, fill=blue!50] (4) at (4,0) {4};
    
    \draw (0) -- (1);
    \draw (0) -- (2);
    \draw (1) -- (2);
    \draw (1) -- (3);
    \draw (2) -- (3);
    \draw (3) -- (4);
\end{tikzpicture}
\end{center}

\textbf{Paso 1:} Inicialización
\begin{itemize}
    \item $result = [-1, -1, -1, -1, -1]$
    \item $result[0] = 0$ (color rojo)
    \item Vértice 0 coloreado con color 0
\end{itemize}

\textbf{Paso 2:} Procesar vértice 1
\begin{itemize}
    \item Vértices adyacentes a 1: [0, 2, 3]
    \item Vértice 0 tiene color 0 $\Rightarrow$ $available[0] = true$
    \item Vértices 2 y 3 no tienen color asignado
    \item Primer color disponible: color 1
    \item $result[1] = 1$ (color azul)
    \item Resetear $available[0] = false$
\end{itemize}

\textbf{Paso 3:} Procesar vértice 2
\begin{itemize}
    \item Vértices adyacentes a 2: [0, 1, 3]
    \item Vértice 0 tiene color 0 $\Rightarrow$ $available[0] = true$
    \item Vértice 1 tiene color 1 $\Rightarrow$ $available[1] = true$
    \item Vértice 3 no tiene color asignado
    \item Primer color disponible: color 2
    \item $result[2] = 2$ (color verde)
    \item Resetear $available[0] = false$, $available[1] = false$
\end{itemize}

\textbf{Paso 4:} Procesar vértice 3
\begin{itemize}
    \item Vértices adyacentes a 3: [1, 2, 4]
    \item Vértice 1 tiene color 1 $\Rightarrow$ $available[1] = true$
    \item Vértice 2 tiene color 2 $\Rightarrow$ $available[2] = true$
    \item Vértice 4 no tiene color asignado
    \item Primer color disponible: color 0
    \item $result[3] = 0$ (color rojo)
    \item Resetear $available[1] = false$, $available[2] = false$
\end{itemize}

\textbf{Paso 5:} Procesar vértice 4
\begin{itemize}
    \item Vértices adyacentes a 4: [3]
    \item Vértice 3 tiene color 0 $\Rightarrow$ $available[0] = true$
    \item Primer color disponible: color 1
    \item $result[4] = 1$ (color azul)
    \item Resetear $available[0] = false$
\end{itemize}

\textbf{Resultado Final Grafo 1:}
\begin{itemize}
    \item Vértice 0 $\Rightarrow$ Color 0 (rojo)
    \item Vértice 1 $\Rightarrow$ Color 1 (azul)
    \item Vértice 2 $\Rightarrow$ Color 2 (verde)
    \item Vértice 3 $\Rightarrow$ Color 0 (rojo)
    \item Vértice 4 $\Rightarrow$ Color 1 (azul)
\end{itemize}
Número de colores usados: 3

\subsection{Ejemplo 2: Grafo 2 con 5 vértices}

\textbf{Grafo de entrada:}
\begin{itemize}
    \item Vértice 0: conectado a [1, 2]
    \item Vértice 1: conectado a [0, 2, 4]
    \item Vértice 2: conectado a [0, 1, 4]
    \item Vértice 3: conectado a [4]
    \item Vértice 4: conectado a [1, 2, 3]
\end{itemize}

\textbf{Representación visual del grafo:}
\begin{center}
\begin{tikzpicture}[node distance=2cm]
    \node[circle, draw, fill=red!50] (0) at (0,0) {0};
    \node[circle, draw, fill=blue!50] (1) at (2,0) {1};
    \node[circle, draw, fill=green!50] (2) at (1,1.5) {2};
    \node[circle, draw, fill=red!50] (3) at (3,1.5) {3};
    \node[circle, draw, fill=blue!50] (4) at (4,0) {4};
    
    \draw (0) -- (1);
    \draw (0) -- (2);
    \draw (1) -- (2);
    \draw (1) -- (4);
    \draw (2) -- (4);
    \draw (3) -- (4);
\end{tikzpicture}
\end{center}

\textbf{Proceso de coloreo:}

\textbf{Paso 1:} $result[0] = 0$ (color rojo)

\textbf{Paso 2:} Vértice 1
\begin{itemize}
    \item Adyacente a 0 (color 0) $\Rightarrow$ $available[0] = true$
    \item Primer color disponible: color 1
    \item $result[1] = 1$ (color azul)
\end{itemize}

\textbf{Paso 3:} Vértice 2
\begin{itemize}
    \item Adyacente a 0 (color 0) y 1 (color 1) $\Rightarrow$ $available[0] = true$, $available[1] = true$
    \item Primer color disponible: color 2
    \item $result[2] = 2$ (color verde)
\end{itemize}

\textbf{Paso 4:} Vértice 3
\begin{itemize}
    \item Adyacente a 4 (sin color) $\Rightarrow$ No hay restricciones
    \item Primer color disponible: color 0
    \item $result[3] = 0$ (color rojo)
\end{itemize}

\textbf{Paso 5:} Vértice 4
\begin{itemize}
    \item Adyacente a 1 (color 1), 2 (color 2), 3 (color 0) $\Rightarrow$ $available[0] = true$, $available[1] = true$, $available[2] = true$
    \item Primer color disponible: color 3
    \item $result[4] = 3$ (color amarillo)
\end{itemize}

\textbf{Resultado Final Grafo 2:}
\begin{itemize}
    \item Vértice 0 $\Rightarrow$ Color 0 (rojo)
    \item Vértice 1 $\Rightarrow$ Color 1 (azul)
    \item Vértice 2 $\Rightarrow$ Color 2 (verde)
    \item Vértice 3 $\Rightarrow$ Color 0 (rojo)
    \item Vértice 4 $\Rightarrow$ Color 3 (amarillo)
\end{itemize}
Número de colores usados: 4

\section{Análisis del Algoritmo}

\subsection{Ventajas del Algoritmo Greedy}
%<*exp4>
\begin{itemize}
    \item \textbf{Eficiencia:} Complejidad lineal en el número de vértices y aristas
    \item \textbf{Simplicidad:} Fácil de implementar y entender
    \item \textbf{Determinismo:} Produce siempre el mismo resultado para el mismo grafo
    \item \textbf{Coloración válida:} Garantiza encontrar una coloración válida
\end{itemize}
%</exp4>

\subsection{Limitaciones}
%<*exp5>
\begin{itemize}
    \item \textbf{No optimalidad:} No garantiza usar el número mínimo de colores
    \item \textbf{Dependencia del orden:} El resultado depende del orden de procesamiento de vértices
    \item \textbf{Coloración subóptima:} Puede usar más colores de los necesarios
\end{itemize}

\textbf{Ejemplo de suboptimalidad:}
Para un grafo bipartito (que requiere solo 2 colores), el algoritmo greedy podría usar hasta $V$ colores dependiendo del orden de procesamiento.
%</exp5>

\subsection{Comparación con Fuerza Bruta}
\begin{itemize}
    \item \textbf{Complejidad:}
    \begin{itemize}
        \item Greedy: $O(V^2 + E)$
        \item Fuerza Bruta: $O(k^V \cdot V \cdot E)$ donde $k$ es el número máximo de colores
    \end{itemize}
    \item \textbf{Ventajas del Greedy:}
    \begin{itemize}
        \item Mucho más eficiente para grafos grandes
        \item Tiempo polinomial vs. exponencial
        \item Práctico para aplicaciones reales
    \end{itemize}
\end{itemize}

\section{Conclusiones}

El algoritmo greedy para coloreo de grafos es una solución eficiente y práctica que garantiza encontrar una coloración válida en tiempo polinomial. Aunque no garantiza optimalidad, su simplicidad y eficiencia lo hacen adecuado para muchas aplicaciones prácticas donde se requiere una coloración válida rápidamente.

Para casos donde se necesita el número mínimo de colores, se requieren algoritmos más sofisticados o técnicas de optimización, pero el algoritmo greedy proporciona una excelente base y punto de partida para el análisis de problemas de coloreo de grafos.

\end{document}
